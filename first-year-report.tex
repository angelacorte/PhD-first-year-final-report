% This is samplepaper.tex, a sample chapter demonstrating the
% LLNCS macro package for Springer Computer Science proceedings;
% Version 2.21 of 2022/01/12
%
\documentclass[runningheads]{llncs}
\usepackage[inline]{enumitem}
\usepackage[utf8]{inputenc}
\usepackage{amsmath}
\usepackage{geometry}
\geometry{a4paper, left=25mm, right=25mm, top=25mm, bottom=25mm}

% \usepackage{fontspec}
%
\usepackage[T1]{fontenc}
% T1 fonts will be used to generate the final print and online PDFs,
% so please use T1 fonts in your manuscript whenever possible.
% Other font encondings may result in incorrect characters.
%
\usepackage{graphicx}
\usepackage{hyperref}
\usepackage{acronym}
\usepackage{cleveref}
% Workaround for: https://tex.stackexchange.com/q/737204/2388
\makeatletter
\AtBeginDocument
{
    \def\ltx@label#1{\cref@label{#1}}%add braces
    \def\label@in@display@noarg#1{\cref@old@label@in@display{#1}}%remove braces
    \def\label@in@mmeasure@noarg#1{%
    \begingroup%
    \measuring@false%
    \cref@old@label@in@display{#1}%remove braces for multline, see https://tex.stackexchange.com/q/737204/2388
    \endgroup
}%
} %
\makeatother

% Used for displaying a sample figure. If possible, figure files should
% be included in EPS format.
%
% If you use the hyperref package, please uncomment the following two lines
% to display URLs in blue roman font according to Springer's eBook style:
%\usepackage{color}
%\renewcommand\UrlFont{\color{blue}\rmfamily}
%
\newenvironment{inlinelist}{\begin{enumerate*}[label=\emph{(\roman*)}]}{\end{enumerate*}}

\acrodef{AP}{Aggregate Programming}
\acrodef{ac}[AC]{Aggregate Computing}
\acrodef{API}{Application Programming Interface}
\acrodef{dsl}[DSL]{domain-specific language}
\acrodef{fc}[FC]{Field Calculus}
\acrodefplural{FC}[FC]{field calculi}
\acrodef{HOFC}{Higher-Order \acl{FC}}
\acrodef{id}[ID]{identifier}
\acrodef{IoT}{Internet of Things}
\acrodef{JVM}{Java Virtual Machine}
\acrodef{scr}[SCR]{self-organising coordination regions}
\acrodef{TOTA}{Tuples On The Air}
\acrodef{VMC}{Vascular Morphogenesis Controller}
\acrodef{cos}[COS]{Collective Operating System}
\acrodef{cas}[CAS]{Collective Adaptive Systems}

\begin{document}

    \begin{titlepage}
        \centering
        \vspace*{2cm}

        {\scshape\Large PhD Programme in Computer Science and Engineering \par}
        \vspace{0.5cm}
        {\scshape\large Cycle XL \par}
        \vspace{0.5cm}

        \rule{\linewidth}{0.4mm} \\ [0.1mm]
        \raisebox{0.2cm}{\rule{\linewidth}{0.8mm}} \\[0.8cm]
        {\huge\bfseries PhD First Year -- Final Report \par}
        \vspace{0.8cm}
        \rule{\linewidth}{0.8mm} \\ [0.1pt]
        \raisebox{0.2cm}{\rule{\linewidth}{0.4mm}} \\[1.5cm]

        % {\Large PhD First Year -- Final Report \par}

        \vspace{1.5cm}

        \noindent
        \begin{minipage}[t]{0.45\textwidth}
            \raggedright
            \textbf{Commission:}\\[0.5cm]
            Prof. Danilo Pianini\\
            Prof. Mirko Viroli\\
            Prof. Enrico Gallinucci
        \end{minipage}%
        \hfill
        \begin{minipage}[t]{0.45\textwidth}
            \raggedleft
            \textbf{PhD Student:}\\[0.5cm]
            Angela Cortecchia
        \end{minipage}

        \vfill

    \end{titlepage}

    \section{Context}\label{sec:context}

    \ac{cas} are ensembles of distributed devices, such as sensors, actuators, and robots,
    that cooperate to achieve global goals under dynamic and unpredictable environmental conditions~\cite{DBLP:conf/huc/Ferscha15}.
    %
    Programming such systems is inherently challenging,
    as it requires addressing issues of scalability, device heterogeneity,
    and the ability to adapt to continuously changing contexts.
    %
    Over the years,
    several approaches have been proposed to alleviate this complexity.
    %
    In particular,
    macroprogramming techniques~\cite{casadei22} aim at raising the level of abstraction,
    allowing developers to express the behavior of the system at a collective level,
    without having to explicitly manage the interactions of each single device.
    %
    A foundational contribution in this direction is the \emph{\ac{fc}}~\cite{JLAMP2019,TOCL2019},
    a formal model for programming collective behaviors through the notion of computational fields,
    which map values over space and time and enable composable definitions of self-organizing processes.

    Building on these ideas,
    \emph{\ac{ac}}~\cite{BealIEEEComputer2015} provides a functional programming approach that supports the compositional development of self-adaptive collective services.
    %
    Through \ac{ac},
    the focus of computation shifts from individual devices to collaborative groups,
    making it possible to express complex adaptive behaviors in a concise and modular way.
    %
    Despite its expressive power,
    however,
    existing aggregate computing approaches typically assume the presence of a single aggregate program running across the distributed system.
    %
    This assumption is often unrealistic in practical scenarios,
    where multiple processes must coexist,
    overlap,
    and evolve in parallel,
    each one involving a different subset of devices and spanning distinct spatial and temporal regions~\cite{EAAI2020-processes}.
    %
    Such situations call for abstractions and mechanisms that are not yet fully addressed in the current state of the art.

    Current existing \ac{ac} approaches are particularly able to adapt to different environmental conditions,
    such as failures, network partitions, and changes in the topology of the network,
    but they lack the ability of maintaining a global state without centralized control that is not monotonic,
    creating subsystems, or structures, that can be used to organize the system in different parts,
    potentially each one with its own behavior and its own goals.
    %
    Some examples of the latter include morphogenetic algorithms:
    these algorithms are inspired by biological processes and are used to create complex structures and patterns
    through simple local interactions among the components of the system~\cite{DBLP:books/daglib/p/Beal12,DBLP:conf/gecco/MorganC13,DBLP:conf/gecco/ZahadatHS17}.
    %
    They can be used in various applications, such as modular robotics, swarm robotics, and self-assembling systems.

    For what it concerns the global state maintenance,
    gossip algorithms are a class of distributed algorithms that allow nodes in a network to exchange information and reach consensus in a decentralized manner.
    %
    However,
    existing gossip algorithms do not guarantee self-stabilization,
    meaning that they may not converge to a correct state after transient faults or changes in the network topology.
    %
    In the \ac{ac} context,
    this limitation has been partially addressed through the introduction of replication mechanism~\cite{PianiniCoordination2016},
    which consists in replicating the gossip process multiple times per node,
    thereby updating the local state and eventually reaching a consistent state.
    %
    However,
    this approach can lead to lagging information in the network before the changes are effectively acknowledged,
    and there is the risk of contaminating the new gossip with obsolete values, resulting in no benefit from the restart.

    To meet these challenges,
    this research project proposes the development of a \textbf{\ac{cos}},
    an execution and management environment for aggregate processes that extends the principles of aggregate computing by borrowing concepts from operating systems.
    %
    Much like a traditional OS manages multiple processes, users,
    and resources on a single centralized machine~\cite{DBLP:journals/csur/TanenbaumR85},
    the COS is envisioned as middleware capable of coordinating multiple aggregate programs distributed in space and time across heterogeneous devices.
    %
    The project investigates how classical notions of operating systems --such as users and permissions, signals and interrupts,
    inter-process communication, and resource management --can be reinterpreted and redefined in a collective and distributed setting.
    %
    In addition,
    this research aims to investigate the feasibility of morphogenetic applications and implementations within the context of \ac{ac}.

    In this perspective,
    users and permissions must be understood as roles and rights distributed across groups of devices and human actors,
    signals and interrupts must be adapted to collective processes where interventions may affect entire spatial regions rather than single execution threads,
    and communication among processes must be reconceived beyond classical shared-memory or message-passing mechanisms,
    to enable coordination among distributed and possibly non-contiguous regions.
    %
    Furthermore,
    the project addresses the challenge of distributed sensing and actuation,
    where heterogeneous devices can be abstracted into logical collective sensors and actuators,
    thereby enabling high-level sensor fusion~\cite{DBLP:journals/arc/Sasiadek02}.

    The proposed \ac{cos} will therefore act as a unifying middleware that manages concurrent aggregate processes,
    provides mechanisms for permissions and communication,
    and supports deployment on heterogeneous infrastructures.
    %
    By integrating recent developments in aggregate programming languages and frameworks,
    such as Collektive, the COS aims to offer a robust, adaptive,
    and general-purpose environment for programming Collective Adaptive Systems.
    %
    The expected contribution is a novel paradigm that bridges aggregate computing with distributed operating system principles,
    enabling more powerful, flexible,
    and reliable collective applications in scenarios such as crowd management, autonomous navigation, and smart cities.

    \section{Methodology and Preliminary Results}\label{sec:methodology}

    The methodology of this research follows a progressive approach,
    meaning that the initial focus has been on developing scenarios to identify requirements and possible challenges
    that the \ac{cos} should address.
    %
    This has been done by applying the \ac{ac} paradigm to implement specific use cases,
    such as morphogenetic algorithms and multi-robot coordination,
    which can benefit from the collective and self-organizing nature of the paradigm.
    %
    Indeed,
    during the development of these use cases,
    a possible improvement of the state of the art has been identified,
    namely the definition of a novel self-stabilizing gossip algorithm within the \ac{ac} paradigm.

    Among the various sub-research lines that compose the overall research project,
    so far the focus and preliminary results have been on the following aspects:
    \begin{itemize}
        \item Implement a morphogenetic algorithm for self-organizing structures through the \ac{ac} paradigm;
        \item Runtime task replanning algorithm for multi-robot missions for large-scale deployments with the \ac{ac} paradigm;
        \item Practical demonstrations of multi-robot coordination based on aggregate computing principles in small-scale scenarios;
        \item Development of a novel self-stabilizing gossip algorithm within the \ac{ac} paradigm.
    \end{itemize}

    \paragraph{Morphogenetic Algorithm.}
    The algorithm \ac{VMC}~\cite{DBLP:conf/gecco/ZahadatHS17},
    inspired by plant branching and nutrient distribution,
    has proven useful for shape formation in modular robotics
    and for resource allocation in hierarchical organizations.
%
    It has been redefined as a field-based computation, \emph{FieldVMC}~\cite{DBLP:conf/acsos/CortecchiaPCC24},
    enabling decentralized and asynchronous emergence of organizational hierarchies
    through local self-organizing interactions.
%
    FieldVMC is topology-independent, inherits scalability, asynchronicity, and self-organization from the \ac{ac} paradigm,
    and, through its functional field-based framework, promotes reuse and composability.
%
    From the perspective of the Collective Operating System project,
    FieldVMC provides a first example of how a complex process can be expressed in terms of field-based computations,
    and how the \ac{ac} paradigm can be used to engineer self-organizing behaviors in a distributed setting.
%
    Moreover,
    the implementation of FieldVMC in the Collektive framework~\footnote{\url{https://github.com/angelacorte/fieldVMC}}
    can be seen as a first step concerning resource management functionalities.
%
    This work is currently under the second round of review for publication in the scientific journal ``Complex \& Intelligent Systems'' (Q1) (\cref{paper:fieldvmc}).

    \paragraph{Multi-Robot Coordination.}
    Another application scenario that has been considered is the coordination of a swarm of robots that must achieve a
    common goal in a shared environment~\footnote{\url{https://github.com/angelacorte/experiments-2025-acsos-robots}}.
%
    In this case,
    the \ac{ac} paradigm has been used to implement a runtime task replanning algorithm for multi-robot missions,
    which allows robots to adapt their plans in real time in response to unpredictable events, such as
    robot failures or environmental changes.
%
    The approach is particularly suitable for large teams deployed in areas with unreliable network infrastructure,
    where centralized control is impractical and network segmentation is frequent.
    %
    This work has been published (\cref{paper:acsos25}) in the proceedings of the ``IEEE International Conference on Autonomic
    Computing and Self-Organizing Systems'' (ACSOS 2025)~\cite{DBLP:conf/acsos/AguzziBCMPPV25}.
%
    From the perspective of the Collective Operating System,
    this work demonstrates how the \ac{ac} paradigm can be used to implement adaptive mechanisms
    that allow multiple processes (i.e., the robots) to coordinate and adapt their behavior in a decentralized manner.

    Another contribution in this direction, presented in the proceedings of the ``Coordination Models and Languages''
    conference (COORDINATION 2025)~\cite{DBLP:conf/coordination/AguzziBBCCDFPV25} (\cref{paper:coordination25}),
    is the design of a toolchain enabling practical multi-robot demonstrations grounded in aggregate computing principles,
    which was validated through a live interactive showcase during the ``European Researchers’ Night''.
    %
    This work shows how a team of mobile robots can be coordinated to form spatial patterns,
    exploiting a camera system and ArUco markers for localization.
    %
    Moreover,
    this project is still ongoing under the name of ``Project Emerge''~\footnote{\url{https://github.com/Project-Emerge}},
    with the goal of showcasing its potential on a larger swarm of robots during the upcoming ``European Researchers’ Night''.

    \paragraph{Self-stabilizing Gossip Algorithm.}

    The proposed self-stabilizing gossip algorithm aims to overcome the main limitations of existing approaches
    by ensuring that the gossip process can recover from transient faults and changes in the network topology,
    while preserving the locality and lightweight nature of gossip.
%
    The algorithm is currently under development and will be integrated into the Collektive framework~\footnote{\url{https://collektive.github.io/}}
    to provide a robust and adaptive mechanism for information dissemination in collective systems.
    %
    The development of this algorithm is a step towards the realization of the Collective Operating System,
    as it provides a foundational building block for resilient coordination and message propagation,
    with \emph{probably} low overhead and high adaptability, in pervasive, self-organizing systems.

    \section{Future Work and Research Plan}\label{sec:future}

    The future work can be divided into short-, mid-, and long-term goals.
%
    In the short term,
    the focus will be on completing the development of the self-stabilizing gossip algorithm
    and integrating it into the Collektive framework.
%
    This will include rigorous testing and validation of the algorithm across different scenarios,
    both in simulations and real-world deployments,
    in order to assess its performance, scalability, and robustness.

    In the mid term,
    the plan is to investigate distributed task scheduling in collaboration with the research group at \emph{Mälardalens University} (Sweden),
    potentially through a period abroad.
%
    This will also involve studying aggregate process mechanisms,
    to evaluate whether they provide a feasible approach for managing such a system.

    As long-term goals,
    the research will focus on the design and implementation of process management mechanisms within the Collective Operating System.
%
    This encompasses defining roles and permissions for collective processes,
    as well as developing signaling and interrupt-handling mechanisms that are appropriate for a distributed and collective context.



    \section{Publications}\label{sec:publications}
    List of published (or already accepted) papers:
    \begin{enumerate}
        \item \emph{A Field-based Approach for Runtime Replanning in Swarm Robotics Missions}~\cite{DBLP:conf/acsos/AguzziBCMPPV25}\label{paper:acsos25}
        \textbf{Abstract:}
        Ensuring mission success for multi-robot systems operating
        in unpredictable environments requires robust mechanisms to react to unpredictable events,
            such as robot failures,
            by adapting plans in real-time.
            %
            Adaptive mechanisms are especially needed for large teams deployed in areas with unreliable network infrastructure,
            for which centralized control is impractical and where network segmentation is frequent.
            %
            This paper advances the state of the art by proposing a field-based runtime task replanning approach grounded in aggregate programming.
            %
            Through this paradigm,
            the mission and the environment are represented by continuously evolving fields,
            enabling robots to make decentralized decisions
            and collectively adapt the ongoing plan.
            %
            We compare our approach with a simple late-stage replanning strategy
            and an oracle-centralized continuous replanner.
            %
            We provide experimental evidence that
            the proposed approach achieves performance close to the oracle if the communication range is sufficient,
            while significantly outperforming the baseline even under sparse communication.
            %
            Additionally, we show that the approach can scale well with the number of robots.

        \item \emph{A Demonstrator for Self-organizing Robot Teams}~\cite{DBLP:conf/coordination/AguzziBBCCDFPV25}\label{paper:coordination25}
        \textbf{Abstract:}
            Aggregate computing is a paradigm with over a decade of investigation and multiple programming frameworks available,
            which proved to be particularly suitable for the simulation of applications in challenging domains such as smart cities and robot swarms.
            %
            This paper introduces a toolchain for practical multi-robot demonstrations based on aggregate computing principles,
            and validates it with a live interactive demo in an open-public event in the context of the European Researchers’ Night.
            %
            More specifically,
            we show how we coordinated a team of mobile robots to form spatial patterns.
            %
            We discuss the practical demonstration performed in an indoor environment,
            which exploits a camera system and ArUco markers for localization.
        %
    \end{enumerate}

    List of submitted papers in phase of review:
    \begin{enumerate}
        \item \emph{FieldVMC: An Asynchronous Model and Platform for Self-organising Morphogenesis of Artificial Structures}\label{paper:fieldvmc} --
        \textbf{Abstract:}
        The VMC is an approach
        to structure development
        inspired by the way plants branch and distribute nutrients.
%
        It has proven useful
        to guide shape formation in modular robotics
        as well as
        resource distribution in hierarchically structured organizations, such as large companies.
%
        In this work,
        we propose FieldVMC: a generalization of VMC,
        founded on the field-based approach known as  aggregate computing,
        which is applicable to arbitrary topologies
        (i.e., undirected graphs rather than trees)
        and supports asynchronous and decentralized execution.
%
%Along this line,
        We redesign VMC as a field-based computation,
        hence enabling the emergence of organizational hierarchies %in a decentralised and asynchronous way,
        out of self-organizing interactions among local entities.
%
        The benefits of our approach are manifold.
%
        Being decentralized and free from topological constraints,
        our approach makes VMC applicable to arbitrary networks;
        being based on a well-known computational model,
        inheriting scalability, asynchronicity, and self-organizing capabilities;
        being implemented in a functional field-based computation framework,
        fostering reuse and composability.
%
        To support our claims, we conduct in-silico quantitative experiments comparing FieldVMC with the original VMC.
        The results demonstrate that FieldVMC is a monotonic extension of VMC, offering
        \begin{inlinelist}
            \item faster convergence,
            and
            \item enhanced capabilities for capturing, analyzing, and engineering novel phenomena.
        \end{inlinelist}
    \end{enumerate}

%    \section{Attended Conferences and Workshops}
%    \begin{enumerate}
%        \item \textbf{Coordination Models and Languages} - 26th International Conference, COORDINATION 2024, Held as Part of the 19th International Federated Conference on Distributed Computing Techniques, DisCoTec 2024, Groningen, The Netherlands, June 17-21, 2024
%        \item \textbf{25th Workshop "From Objects to Agents"}, Bard (Aosta), Italy, July 8-10, 2024
%        \item \textbf{IEEE International Conference on Autonomic Computing and Self-Organizing Systems}, ACSOS 2024, Aarhus, Denmark, September 16-20 2024
%        \item \textbf{11th Workshop on Self-Improving Systems Integration}, SISSY 2024, Held as Part of ACSOS 2024, Aarhus, Denmark, September 16-20 2024
%        \item \textbf{2nd International Workshop on Artificial Intelligence for Autonomous computing Systems}, AI4AS 2024, Held as Part of ACSOS 2024, Aarhus, Denmark, September 16-20 2024
%        \item \textbf{28th International Symposium on Distributed Simulation and Real Time Applications}, DS-RT 2024, Urbino, Italy, October 7-9 2024
%    \end{enumerate}

    \section{Doctoral Schools}\label{sec:doctoral-schools}
    \begin{enumerate}
        \item \textbf{SPAICERAISE, The "International Doctoral School" For The Space Sector}, SPACERAISE 2025, L'Aquila, Italy, May 12-16 ,2025
        \item \textbf{Bertionoro International Spring School 2025}, BISS 2025, Bertinoro, Italy, May 19-23 ,2025
        \item \textbf{5th International Software Engineering Summer School}, SIESTA 2025, Lugano, Switzerland, August 27-29 ,2025
    \end{enumerate}

    \section{Teaching Tutor}
    \begin{enumerate}
        \item \textbf{Architetture Degli Elaboratori} -- Computer Science and Engineering Bachelor Degree
%        \item \textbf{Software Engineering (Modulo 1)} - Digital Transformation Management Master Degree
    \end{enumerate}

    \section{PhD Courses}

    \begin{enumerate}
        \item \textbf{BISS 2025 -- Bertinoro International Spring School -- \textit{Quantum Computing and Software Engineering}}
        \begin{itemize}
            \item Prof: Shaukat Ali -- Simula Research Laboratory, Norway
            \item Proposed CD: 3.6 (12 hours)
            \item Period: May 2025
            \item Exam: Done, CDs with evaluation
        \end{itemize}
        \item \textbf{BISS 2025 -- Bertinoro International Spring School -- \textit{Building Language Models: A Practical Introduction}}
        \begin{itemize}
            \item João Monteiro -- Research Scientist at Autodesk
            \item Proposed CD: 3.6 (12 hours)
            \item Period: May 2025
            \item Exam: Done, CDs with evaluation
        \end{itemize}
        \item \textbf{BISS 2025 -- Bertinoro International Spring School -- \textit{Service Engineering: From Design to Operation}}
        \begin{itemize}
            \item Prof: Pablo Fernandez -- University of Seville, Spain
            \item Proposed CD: 3.6 (12 hours)
            \item Period: May 2025
            \item Exam: Done, CDs with evaluation
        \end{itemize}
        \item \textbf{Service Orchestration and Industrial IoT Platforms for Industry 4 and 5.0 environments}
        \begin{itemize}
            \item Prof: Riccardo Venanzi -- Università di Bologna
            \item Proposed CD: 2.4 (12 hours)
            \item Period: January 2025
            \item Exam: Doing (lectures already attended), CDs with evaluation
        \end{itemize}
        \item \textbf{Containerisation and Orchestration for Research Reproducibility}
        \begin{itemize}
            \item Prof: Giovanni Ciatto -- Università di Bologna
            \item Proposed CD: 2.4 (12 hours)
            \item Period: December 2024
            \item Exam: Done, CDs with evaluation
        \end{itemize}
        \item \textbf{Multi-platform Programming for Research-Oriented Software}
        \begin{itemize}
            \item Prof: Giovanni Ciatto -- Università di Bologna
            \item Proposed CD: 2.4 (12 hours)
            \item Period: December 2024
            \item Exam: Done, CDs with evaluation
        \end{itemize}
        \item \textbf{Toward Next-Generation Networks -- 5G and O-RAN implementations}
        \begin{itemize}
            \item Prof: Domenico Scotece -- Università di Bologna
            \item Proposed CD: 0.48 (12 hours)
            \item Period: November 2024
            \item Exam: Presence, CDs with evaluation
        \end{itemize}
        \item \textbf{Human-AI Interaction}
        \begin{itemize}
            \item Prof: Giovanni Delnevo -- Università di Bologna
            \item Proposed CD: 2.4 (12 hours)
            \item Period: April 2025
            \item Exam: Done, CDs with evaluation
        \end{itemize}
        \item \textbf{Robust and Reproducible Research}
        \begin{itemize}
            \item Prof: Federico Ruggeri -- Università di Bologna
            \item Proposed CD: 3.2 (16 hours)
            \item Period: April 2025
            \item Exam: Done, CDs with evaluation
        \end{itemize}
        \item \textbf{Data Visualization for Researchers: theory and new approaches to enhance communication and dissemination}
        \begin{itemize}
            \item Prof: Chiara Ceccarini -- Università di Bologna
            \item Proposed CD: 2.4 (12 hours)
            \item Period: April 2025
            \item Exam: Done, CDs with evaluation
        \end{itemize}

    \end{enumerate}

    \section{Attended Dissemination Events}\label{sec:attended-dissemination-events}
    \begin{enumerate}
        \item \textbf{PhD Symposium in CSE} -- Computer Science and Engineering Department, Università di Bologna, Italy, May 30, 2025
    \end{enumerate}


    \bibliographystyle{alpha}
    \bibliography{bibliography}

\end{document}